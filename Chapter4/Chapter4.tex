%This is the second chapter of the dissertation

%The following command starts your chapter. If you want different titles used in your ToC and at the top of the page throughout the chapter, you can specify those values here. Since Columbia doesn't want extra information in the headers and footers, the "Top of Page Title" value won't actually appear.
\pagestyle{plain}

\chapter[Closed-loop solutions for silicon photonic ring resonators][Top of Page Title]{Closed-loop solutions for silicon photonic ring resonators}

This high thermo-optic susceptibility, on one hand, can be an opportunity as it allows for effective manipulation of the light in thermo-optic modulators \cite{nedeljkovic2014mid,gautam2012single,wang2003soi} and optical switches \cite{suzuki2017broadband,li2016silicon,liu2016two}, for example. On the other hand, for very narrow-band optical devices such as ring resonator filters, thermal susceptibility is detrimental, and  requires  accurate monitoring and control of the temperature to obtain desired behavior and performance \cite{gazmanautomated,zortman2013bit,mahendra2017multiwavelength,padmaraju2014resolving,padmaraju2012thermal}. For example, the resonant frequency of a silicon ring resonator is shifted by $\sim$9 Ghz ($\sim$0.07 nm) for each degree Kelvin change in the temperature of the waveguide \cite{masood2013comparison,pintus2016optimization}. A single degree of temperature variation is therefore sufficient to create significant spectral distortion  of on-off keying (OOK) signals for dense wavelength-division multiplexing (DWDM) systems that employ add-drop ring filters at the receiver \cite{bahadori2016crosstalk,bahadori2016energy,sun2015single}. 

Different mechanisms of thermal control and compensation have been proposed to improve tolerance of narrow bandwidth devices to ambient temperature variations. The use of polymer cladding that counteracts the thermo-optic effect of silicon has been proposed \cite{guha2013athermal}; however, this approach requires a specific width of the waveguide ($\sim$306 nm \cite{padmaraju2014resolving}) and some post fabrication processes that are not CMOS compatible. Polymers have also been used in the design of low-power thermo-optic switches \cite{niu2017optimized}. The most popular solution leverages active thermal control and consists of sending an electrical current through external but closely integrated metallic \cite{atabaki2010optimization} or doped-silicon based micro-heaters to induce ohmic heating \cite{pintus2016optimization}. Fabrication constraints require metallic heaters to be located above the waveguide and doped-silicon heaters to be located next to the waveguide. In both cases, the generated heat diffuses mainly through the silica (SiO$_2$) cladding layer to reach the silicon waveguide \cite{masood2013comparison,niu2017optimized}. Solutions where the waveguide itself is turned into a heater by tapering it to a wider width and doping it locally have also been proposed in \cite{watts2009adiabatic, derose2011low}, and \cite{watts2013adiabatic}. An approach consisting of a PN-doped waveguide driven with a reversed voltage to its breakdown to produce Ohmic heat right inside it has also been proposed by Li \textit{et al}. \cite{li2014fast}. Heat diffusion by means of external heaters, however, permits a separation between the design of the waveguide (choice of optical mode and polarization) and the design of the heating element.


\section{Introduction}

By using the asterisk to start a new section, I keep the section from appearing in the table of contents.
If you want your sections to be numbered and to appear in the table of contents, remove the asterisk.
\section{Sensitivities in Micro Ring Resonators}
\section{Micro-ring resonators sensitivity issues: fabrication variations, thermal sensitivity and self-heating}

By using the asterisk to start a new section, I keep the section from appearing in the table of contents.
If you want your sections to be numbered and to appear in the table of contents, remove the asterisk.

